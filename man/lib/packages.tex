\usepackage{enumitem}
%\SetLabelAlign{parright}{\parbox[t]{\labelwidth}{\raggedleft#1}}
\setlist[description]{ style = multiline
                     , labelwidth = 70pt
                     , itemindent=\parindent
                     , leftmargin=7em
                     , labelindent=1em,leftmargin=!
                     , labelindent=!
                     %, align=parright
                     }
\usepackage{adjustbox}      % for minipage: inserting figures in enumerated lists
\usepackage{listings}     % for code typing
\usepackage{xifthen}
\usepackage{textcomp} % for \textregistered
\usepackage[numbers,comma,sort,compress]{natbib}
\usepackage{amsfonts,amssymb,amsbsy,amsmath}
\usepackage{labman}
\usepackage{scrextend}
%\usepackage{amsmath} %,times,mtpro2} %% times package needed only to load TNR text font}
\usepackage{nicefrac}
\usepackage{color}
\usepackage{graphicx}
\usepackage{caption}
\usepackage{subcaption}
\usepackage[hyphens]{url} % The option hyphens breaks the very long urls into multiple lines to obey the margins. See: https://tex.stackexchange.com/questions/3033/forcing-linebreaks-in-url
\usepackage{fancyhdr}
\usepackage[usenames,dvipsnames]{xcolor}
\usepackage{perpage}
%\usepackage{multirow}

\usepackage{hyperref} % PDF links
\hypersetup{
    colorlinks=true,
    linkcolor=blue,
    filecolor=magenta,
    urlcolor=cyan,
    pdftitle={Sharelatex Example},
    bookmarks=true,
    %pdfpagemode=FullScreen,
}

\usepackage{setspace} % for switching between double/single space in document
\usepackage[version=4]{mhchem} % for chemistry notations
\usepackage[theorems]{tcolorbox}
\usepackage{enumitem}% http://ctan.org/pkg/enumitem
\usepackage{boxedminipage}
\usepackage{wrapfig}
\usepackage[labelfont=bf]{caption}
%\usepackage[a4paper, margin=1in]{geometry}
\usepackage{gensymb} % \degree command
\usepackage{tabularx}
\usepackage{array} % used in tabular env for left alignment of text >{\raggedright}
\usepackage{import} % for nested input files
\usepackage{chemfig}
%\setlist[itemize]{leftmargin=\parindent, labelindent=\parindent}
%\setlist[itemize]{align=left}
%\usepackage[T1]{fontenc}   % used for font consistency across multiple tex subfiles. This arised when I used the main files generated by overleaf with tex subfiles (chapters) generated by notepad++.
%\MakePerPage{footnote} % resets the footnote symbol counter for each page

% Begin commands for sorted itemize
\usepackage{datatool}% http://ctan.org/pkg/datatool
\newcommand{\sortitem}[1]{%
  \DTLnewrow{list}% Create a new entry
  \DTLnewdbentry{list}{description}{#1}% Add entry as description
}
\newenvironment{sortedlist}{%
  \DTLifdbexists{list}{\DTLcleardb{list}}{\DTLnewdb{list}}% Create new/discard old list
}{%
  \DTLsort{description}{list}% Sort list
  \begin{itemize}%
    \DTLforeach*{list}{\theDesc=description}{%
      \item \theDesc}% Print each item
  \end{itemize}%
}
% End commands for sorted itemize

%\usepackage{mymacros}
%\usepackage{nextpage} % Provides \clearpage \newpage variants that guarantee to end up on even/odd num­bered pages
% NOTE: If used with the epigraph package, then nextpage.sty must be loaded first.
% Example uses:
% \cleartooddpage                         % same as \cleardouble page
% \cleartooddpage[\thispagestyle{empty}]  % No headings on the skipped page
% \cleartoevenpage                        % go to next even numbered page
% Next example puts text on a skipped page
% \cleartoevenpage[\vspace*{\hfill}THIS PAGE LEFT BLANK\vspace*{\hfill}]

%\usepackage[linewidth=1pt]{mdframed}
% setting the margins of page
%\usepackage[top=3cm,right=3cm,bottom=2.5cm,left=2.5cm]{geometry}
%\usepackage{rotate}
%\usepackage[T1]{fontenc}   % used for font consistency across multiple tex subfiles. This arised when I used the main files generated by overleaf with tex subfiles (chapters) generated by notepad++.
%\MakePerPage{footnote} % resets the footnote symbol counter for each page
\usepackage{pifont,latexsym,ifthen,theorem,rotating,calc}
\usepackage{multirow}
\newenvironment{rcases}
  {\left.\begin{aligned}}
  {\end{aligned}\right\rbrace}
\newtheorem{thm}{Theorem}%[section]
\newtheorem{illust}[thm]{Illustration}
\theoremstyle{remark}
\newtheorem{rem}{Remark}

% Begin commands for table in lecture 8
\usepackage{array}
\newcommand{\rstop}{\textcolor{red}{STOP}}
\makeatletter
\newcommand{\thickhline}{%
    \noalign {\ifnum 0=`}\fi \hrule height 2pt
    \futurelet \reserved@a \@xhline
} 