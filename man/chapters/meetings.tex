\chapter{Meetings}
\label{sec:meetings}

The names of the attendants who suggest or advocate specific ideas in each meeting follows (in boldface font) the description of the idea itself.

%%%%%%%%%%%%%%%%%%%%%%%%%%%%%%%%%%%%%%%%%%%%%%%%%%%%%%%%%%%%%%%%%%%%%%%%%%%%%%%%%%%%%%%%%%%%%%%%%%%%%%%%%%%%%%%%%%%%%%%%%%%%%%%%%%%%
%%%%%%%%%%%%%%%%%%%%%%%%%%%%%%%%%%%%%%%%%%%%%%%%%%%%%%%%%%%%%%%%%%%%%%%%%%%%%%%%%%%%%%%%%%%%%%%%%%%%%%%%%%%%%%%%%%%%%%%%%%%%%%%%%%%%

\section{1st Joint Meeting: Monday, 10:00-11:15, Sep 10 2018}
\label{sec:meetings:20180910}

    \people
        \amirs, \amirf, \leili, \daniel
        \newpar

    \agenda{
        \item Discuss the development of Data 2401: Intro Scientific Computation course and other Data courses in the broader scope.
    }

    \topics{

        \item
            {\bf ``Intro Data Science'' Curriculum Development:}
            \begin{itemize}
                \item
                    \li from the Mathematics department has already developed the course material for Intro Data Science and is currently teaching the course.
                \item
                    {\bf Course structure:} \li is using Python instead of Excel (as in the original plan) for this course. This may be a good idea since Excel is rarely used a Data Science software tool in the community (\leili). However, to ensure the original goals of the course are met, all relevant Python functionalities corresponding to the equivalent Excel functionalities originally desired as an outcome of this course could be taught instead, so that students learn the same Data Science skills with Python instead of Excel (\amirs).
                \item
                    Instead, she recommends adding SQL to the curriculum, which is currently missing.
                \item
                    \li is implementing her lecture notes in Jupyter notebooks and posts them on Blackboard.
                \item
                    \li is teaching a combination of Data Science materials, from simple statistical and data manipulation tasks, to Python programming, to simple concepts in Machine Learning like K-means, Nearest-Neighbor, Random Forests, ... (the course material will be soon shared with all). One advantage of such heterogenous course contents is that it can give the students an overview of the essential skills that the students would expect to learn throughout their degree program (\amirf). In particular, students in the subsequent The It can One idea is to get the blackbox tools from Intro Data Science course to implement in Scientific computing.
            \end{itemize}

        \item
            SQL currently seems to be missing in the list of courses to be developed/taught in the program. As an essential tool for manipulating large dataset, we may have to redesign the existing or add new course to include it in the curriculum (\leili).

        \item
            {\bf The program website design:} The program's website has to be setup as soon as possible (\dean, \leili, \amirf, \amirs). This will be the discussion of the next meeting. Some initial guidance from the more senior college-level administrators or team-members will be certainly needed (via meetings) to ensure the website development follows the grand goals of the program (\amirf, \amirs).

            A good website development and design will be likely best achieved by hiring a dedicated web developer (\amirf, \amirs). 

        \item
            A shared repository will be setup for all the meetings' material (such as \li's course documents) that will be accessible to everyone in the team (\amirf). Possibilities at the moment include Dropbox (\leili), UTBox (\li, \amirf, \amirs, \leili), or GitHub (\amirf).

        \item
            {\bf Possible Teaching Platforms:} 
            Several teaching platforms were discussed between \amirf, \amirs, and \leili over the summer:
            \begin{enumerate}
                \item
                    The Texas Advanced Computing Resources for Data Science (\amirs):
                \item
                    Third-party clouds, such as OpenStack, Google, ... (\leili): Google provides it for free for teaching purposes.
                \item
                    Amazon cloud, or alternatively, a dedicated UTA internal cloud server (\amirf).
            \end{enumerate}

        \item
            Take advantage of UTA student help program (library) (\leili).

        \item
            The Intro Scientific Computing II should focus more on advanced high-performance programming languages (C/C++/Fortran) (\daniel, \amirs).

        \item
            {\bf Exams / Exercise / Homework:} How can plagiarism be minimized? (\amirf) Still open-ended.

            
    }

%%%%%%%%%%%%%%%%%%%%%%%%%%%%%%%%%%%%%%%%%%%%%%%%%%%%%%%%%%%%%%%%%%%%%%%%%%%%%%%%%%%%%%%%%%%%%%%%%%%%%%%%%%%%%%%%%%%%%%%%%%%%%%%%%%%%
%%%%%%%%%%%%%%%%%%%%%%%%%%%%%%%%%%%%%%%%%%%%%%%%%%%%%%%%%%%%%%%%%%%%%%%%%%%%%%%%%%%%%%%%%%%%%%%%%%%%%%%%%%%%%%%%%%%%%%%%%%%%%%%%%%%%
