\section{Introduction}
\label{sec:intro}

    The Computational Data Science (CDS) Lab at The University of Texas at Arlington (UTA) carries out research in a wide range of scientific disciplines, including Biomedicine, Biophysics, Astrophysics, Mathematical Modeling, and Scientific Software Development, all of which require intensive usage and development of Computational and Data Science methodologies and algorithms.

    While the research carried out in the CDS Lab is purely theoretical / computational, we collaborate frequently and welcome new collaborations with experimental groups and clinicians across the UTA campus, DFW Metroplex, as well as nationally / internationally.

    The CDS Lab is proudly a member of the College of Science at The University of Texas at Arlington and is heavily involved in the development of the Data Science Program as part of the UTA's Strategic Plan $\vert$ 2020.

    This manual is written for the current, new, or prospective members of the CDS Lab to help them communicate, collaborate, and carry out research at the highest levels of efficiency and scientific integrity possible.

\section{Current and Past Members}
\label{sec:members}

    For information about current and past members, as well as the Lab PI, please see \url{https://cdslab.org/people/}.

\section{Communication}
\label{sec:communication:instant}

\subsection{Instant Communication}
\label{sec:communication:instant}

    \begin{itemize}

        \item
            {\bf Slack:} \url{https://cdslaborg.slack.com} \\
            Slack is the primary tool for instant communication among members of the CDS Lab. All lab members should download the Slack Application on their mobile device, and use it for instant communication with individual or all members of the lab, including the PI (Amir). Non-urgent lengthy time-consuming issues can be discussed via email or in person.

    \end{itemize}

\subsection{Other Communications}
\label{sec:communication:other}

    The CDS Lab has webpages on multiple media platforms, including,

    \begin{itemize}

        \item
            {\bf Medium:} \url{https://medium.com/cdslab} \\
            This is a social platform that we use mostly for communicating medium-to-long pieces of scientific research or educational material, performed by the lab members, in the form of blog-posts with the world. All lab members are welcome and encouraged to contribute posts to this page. Medium is a highly-regarded platform for advertising your work to the entire world. Writing an elegant readable article on this page that could attract public's attention, could significantly boost your profile as an expert in your field.

        \item
            {\bf Facebook:} \url{https://fb.me/cdslab} \\
            Facebook is mostly used for communicating the lab news and articles with the Lab members as well as public community on Facebook. You can follow the page by clicking on the link provided above.

        \item
            {\bf Twitter:} \url{https://twitter.com/cdslaborg} \\
            Similar to Facebook, Twitter is mostly used for communicating the lab news and articles with the Lab members as well as the public community on Twitter. You can follow the page by clicking on the link provided above.

        \item
            {\bf Instagram:} \url{https://www.instagram.com/cdslaborg} \\
            Similar to Facebook and Twitter, Instagram is mostly used for communicating the lab news and articles with the Lab members as well as the public community on Instagram. Obviously, any news that appears on Instagram has to be followed by an illustration or photo that would serve as a visual explanation of the news.

        \item
            {\bf GitHub:} \url{https://github.com/cdslaborg} \\
            GitHub is a highly popular code repository that we use for keeping track of changes in our scientific codes and data as well as for sharing our codes and data processing pipelines with the entire world, permanently, once it is complete. As a member of a data-intensive lab, you are expected to have a GitHub account. There are also other, almost equally-capable online code repositories available, such as Bitbucket. However, at the moment GitHub is the primary code repository of the lab.

        \item
            {\bf Dropbox:} \url{https://dropbox.com/} \\
            Like GitHub, we often use Dropbox as way of communicating all digital information such as code, data, articles, manuscripts, but only for our own internal usage (unlike Git and GitHub which are used for professional version controlling of the codes, as well as for sharing scientific research pipelines and information with the entire world. If you don't have a Dropbox account yet, please let Amir know as soon as possible to send you an invitation email.

    \end{itemize}


\section{Introduction}
\label{sec:intro}
